\documentclass[10pt]{beamer}

% ================== PACKAGES ==================
\usepackage{graphicx}
\usepackage{amsmath}
\usepackage{hyperref}
\usepackage{xcolor}

% ================== THEME ==================
\usetheme{Madrid}

% ----- GREY TITLE BAR & FOOTER ONLY -----
\definecolor{mygrey}{RGB}{120,120,120}

% Title bar (frame title)
\setbeamercolor{frametitle}{fg=white,bg=mygrey}

% Footer (author / title / date line)
\setbeamercolor{author in head/foot}{fg=white,bg=mygrey}
\setbeamercolor{title in head/foot}{fg=white,bg=mygrey}
\setbeamercolor{date in head/foot}{fg=white,bg=mygrey}

% Keep all slide text black
\setbeamercolor{normal text}{fg=black,bg=white}
\setbeamercolor{structure}{fg=black}

% ================== TITLE DETAILS ==================
\title[Landslide Detection]{Landslide Detection Using Deep Learning and Multispectral Satellite Images}
\author{Geethanjali Javvadi}
\institute{Department of ECE \\ GVPCEW Madhurawada}
\date{\today}

% ================== DOCUMENT ==================
\begin{document}
	
	% ---------- TITLE SLIDE ----------
	\begin{frame}
		\titlepage
	\end{frame}
	% ---------- TABLE OF CONTENTS ----------
	\begin{frame}{Table Of Contents}
		\tableofcontents
	\end{frame}
	
	% ---------- PROBLEM STATEMENT ----------
	\section{Introduction}
	\begin{frame}{Problem Statement}
		\begin{itemize}
			\item Landslides are among the most destructive natural disasters.
			\item They cause significant loss of:
			\begin{itemize}
				\item Human life
				\item Infrastructure
				\item Natural resources
			\end{itemize}
			\item Accurate landslide detection is challenging due to:
			\begin{itemize}
				\item Complex terrain conditions
				\item Variations in soil, vegetation, and rainfall
			\end{itemize}
			\item Traditional landslide identification methods are:
			\begin{itemize}
				\item Manual and time-consuming
				\item Costly and not scalable to large areas
			\end{itemize}
		\end{itemize}
	\end{frame}
	% ---------- NEED FOR AUTOMATION ----------
	\begin{frame}{Need for Automation}
		\begin{itemize}
			\item Manual landslide monitoring is not feasible for large-scale regions.
			\item Rapid disaster response requires timely and accurate detection.
			\item Satellite imagery provides:
			\begin{itemize}
				\item Wide-area and remote coverage
				\item Frequent temporal observations
				\item Rich spatial and spectral information
			\end{itemize}
			\item Deep Learning enables:
			\begin{itemize}
				\item Automatic feature extraction
				\item Improved detection accuracy
				\item Scalable and reliable analysis
			\end{itemize}
		\end{itemize}
	\end{frame}
	% ---------- PREVIOUS WORK ----------
	\section{Previous Work}
	\begin{frame}{Previous Work}
		\begin{itemize}
			\item Early landslide detection relied on:
			\begin{itemize}
				\item Manual field surveys
				\item Visual interpretation of aerial and satellite images
			\end{itemize}
			\item Traditional machine learning approaches:
			\begin{itemize}
				\item Support Vector Machines (SVM)
				\item Random Forests
				\item Logistic Regression
			\end{itemize}
			\item Deep learning-based methods:
			\begin{itemize}
				\item Convolutional Neural Networks (CNN)
				\item ResNet and DenseNet architectures
				\item U-Net for pixel-level landslide segmentation
			\end{itemize}
		\end{itemize}
	\end{frame}
	% ---------- LIMITATIONS OF PREVIOUS WORK ----------
	\begin{frame}{Limitations of Previous Work}
		\begin{itemize}
			\item Landslide datasets are highly imbalanced:
			\begin{itemize}
				\item Very few landslide samples compared to non-landslide samples
			\end{itemize}
			\item Existing deep learning models tend to:
			\begin{itemize}
				\item Overfit on limited landslide data
				\item Perform poorly on unseen regions
			\end{itemize}
			\item Limited utilization of multispectral satellite information:
			\begin{itemize}
				\item Many methods rely mainly on RGB bands
			\end{itemize}
			\item Fully connected classifiers are biased toward majority classes.
		\end{itemize}
	\end{frame}
	% ---------- NEED FOR IMPROVEMENT ----------
	\begin{frame}{Need for Improvement}
		\begin{itemize}
			\item An effective landslide detection system should:
			\begin{itemize}
				\item Handle severe class imbalance explicitly
				\item Improve generalization on unseen geographical regions
			\end{itemize}
			\item Proper utilization of multispectral satellite data is required:
			\begin{itemize}
				\item To capture terrain, soil, and vegetation variations
			\end{itemize}
			\item Robust classification methods are needed to:
			\begin{itemize}
				\item Reduce bias toward majority classes
				\item Enhance decision boundaries
			\end{itemize}
			\item These requirements motivate the proposed deep learning framework.
		\end{itemize}
	\end{frame}
	% ---------- DATASET DESCRIPTION ----------
	\section{Dataset Preparation}
	\begin{frame}{Dataset Description}
		\begin{itemize}
			\item Multispectral satellite image dataset used for landslide detection.
			\item Dataset source:
			\begin{itemize}
				\item Downloaded from the \textbf{Zindi Africa} platform
				\item Obtained by registering for the \textbf{Landslide Detection Competition}
			\end{itemize}
			\item Each data sample consists of:
			\begin{itemize}
				\item Image size: $64 \times 64$ pixels
				\item 12 spectral bands per image
			\end{itemize}
			\item Data format:
			\begin{itemize}
				\item Stored as \texttt{.npy} files
				\item Represents embeddings of landslide and non-landslide regions
			\end{itemize}
			\item Dataset contains two classes:
			\begin{itemize}
				\item Landslide
				\item Non-landslide
			\end{itemize}
			\item Significant class imbalance is observed in the dataset.
		\end{itemize}
	\end{frame}
	% ---------- DATASET CHALLENGES ----------
	\begin{frame}{Challenges in the Dataset}
		\begin{itemize}
			\item Severe class imbalance:
			\begin{itemize}
				\item Landslide samples are much fewer than non-landslide samples
			\end{itemize}
			\item High visual similarity between:
			\begin{itemize}
				\item Landslide regions
				\item Bare soil and rocky terrain
			\end{itemize}
			\item Presence of noise due to:
			\begin{itemize}
				\item Vegetation cover
				\item Atmospheric conditions
			\end{itemize}
			\item Variations in terrain and geographical conditions across regions.
		\end{itemize}
	\end{frame}
	
	% ---------- DETAILED METHODOLOGY ----------
	\section{Methodology}
	\begin{frame}{Detailed Methodology}
		\begin{itemize}
			\item \textbf{Input Data:}
			\begin{itemize}
				\item Multispectral satellite images with 12 spectral bands
				\item Images resized and normalized for uniform processing
			\end{itemize}
			
			\item \textbf{Handling Class Imbalance:}
			\begin{itemize}
				\item Offline SMOTE (Synthetic Minority Over-sampling Technique)
				\item Generates synthetic landslide samples to balance the dataset
			\end{itemize}
			
			\item \textbf{Feature Extraction:}
			\begin{itemize}
				\item EfficientNetV2-Large used as backbone CNN
				\item Modified input layer to support 12-channel images
				\item Deep features extracted instead of direct classification
			\end{itemize}
			
			\item \textbf{Regularization and Generalization:}
			\begin{itemize}
				\item Online data augmentation using MixUp and CutMix
				\item Reduces overfitting and improves robustness
			\end{itemize}
			
			\item \textbf{Final Classification:}
			\begin{itemize}
				\item Support Vector Machine (SVM) applied on extracted features
				\item Improves decision boundary and minority class performance
			\end{itemize}
		\end{itemize}
	\end{frame}
% ---------- MODEL ARCHITECTURE (DETAILED OVERVIEW) ----------
\begin{frame}{Deep Learning Model Architecture}
	\begin{itemize}
		\item \textbf{EfficientNetV2-Large} is chosen as the backbone CNN due to its strong performance on image classification tasks.
		\item It follows a \textbf{compound scaling strategy}:
		\begin{itemize}
			\item Balances network depth, width, and resolution
			\item Achieves higher accuracy with fewer parameters
		\end{itemize}
		\item Compared to traditional CNNs:
		\begin{itemize}
			\item Faster convergence during training
			\item Better generalization on unseen data
		\end{itemize}
		\item Well-suited for large-scale and high-dimensional image data.
	\end{itemize}
\end{frame}

% ---------- MODEL ARCHITECTURE (MULTISPECTRAL ADAPTATION) ----------
\begin{frame}{Model Adaptation for Multispectral Data}
	\begin{itemize}
		\item Original EfficientNetV2 is designed for 3-channel RGB images.
		\item In this work, the input layer is \textbf{modified to accept 12 channels}:
		\begin{itemize}
			\item Allows effective utilization of multispectral satellite information
			\item Captures terrain, vegetation, and soil characteristics
		\end{itemize}
		\item The classification head is removed to:
		\begin{itemize}
			\item Extract high-level feature embeddings
			\item Avoid bias from fully connected layers
		\end{itemize}
		\item Extracted deep features are used for robust SVM-based classification.
	\end{itemize}
\end{frame}

	
	% ---------- MODEL ARCHITECTURE (DIAGRAM) ----------
% ---------- EFFICIENTNETV2-LARGE ARCHITECTURE (UNIT BLOCKS PLACEHOLDER) ----------
\begin{frame}{EfficientNetV2-Large Architecture with Unit Blocks}
	\begin{center}
		% Replace the filename below with your architecture screenshot
		\includegraphics[width=1\linewidth]{efficientnetv2_architecture.png}
	\end{center}
	
	\vspace{0.3cm}
	\begin{itemize}
		\item Architecture consists of stacked \textbf{unit blocks}:
		\begin{itemize}
			\item Stem Convolution Block
			\item Fused-MBConv Blocks
			\item MBConv Blocks
			
		\end{itemize}
		
	\end{itemize}
\end{frame}
% ---------- HIGH-LEVEL ARCHITECTURE OF PROPOSED FRAMEWORK ----------
\begin{frame}{High-Level Architecture of Proposed Deep Learning Framework}
	\begin{center}
		% Replace with your high-level architecture diagram screenshot
		\includegraphics[width=1\linewidth]{model_architecture.png}
	\end{center}


\end{frame}
% ---------- DATA AUGMENTATION & SMOTE (DETAILED) ----------
\section{Data Augmentation and SMOTE}
\begin{frame}{Data Augmentation and SMOTE}
	\begin{itemize}
		\item The landslide detection dataset exhibits \textbf{severe class imbalance}:
		\begin{itemize}
			\item Landslide samples are significantly fewer than non-landslide samples
			\item This bias negatively affects model learning and minority class prediction
		\end{itemize}
		
		\item \textbf{Offline SMOTE (Synthetic Minority Over-sampling Technique)} is employed:
		\begin{itemize}
			\item Generates synthetic samples for the landslide class
			\item Operates in feature space to avoid simple duplication
			\item Improves class balance before training begins
		\end{itemize}
		
		\item \textbf{Online Data Augmentation} is applied during training:
		\begin{itemize}
			\item \textbf{MixUp}: Combines pairs of images and labels to smooth decision boundaries
			\item \textbf{CutMix}: Replaces regions of one image with another to improve localization
			\item \textbf{Intensity transformations}: Handle illumination and spectral variations
			\item \textbf{Geometric transformations}: Improve invariance to rotation and flipping
		\end{itemize}
		
		\item The combined use of SMOTE and augmentation:
		\begin{itemize}
			\item Reduces overfitting
			\item Enhances model robustness
			\item Improves minority (landslide) class performance
		\end{itemize}
	\end{itemize}
\end{frame}
% ---------- SMOTE VISUALIZATION ----------
\begin{frame}{SMOTE Visualization and Class Distribution}
	\begin{figure}
		\centering
		% ----------- (a) SMOTE SAMPLE GENERATION -----------
		\includegraphics[width=0.9\linewidth]{smote_generation_example.png}
		\caption*{\textbf{(a)} Left: Anchor images \quad Middle: Nearest neighbors \quad Right: SMOTE-generated samples\newline \textbf{(b)} Class distribution on the training subset after applying the SMOTE algorithm}
	\end{figure}
\end{frame}
% ---------- TRAINING SETUP ----------
\section{Training Setup}
\begin{frame}{Training Setup}
	\begin{itemize}
		\item \textbf{Backbone Network:} EfficientNetV2-Large
		\item \textbf{Input:}
		\begin{itemize}
			\item Multispectral images of size $64 \times 64$
			\item 12 spectral channels
		\end{itemize}
		\item \textbf{Training Configuration:}
		\begin{itemize}
			\item Optimizer: Adam
			\item Initial Learning Rate: $3 \times 10^{-4}$
			\item Learning Rate Scheduler: Cosine Annealing
			\item Batch Size: 8
			\item Number of Epochs: 50
		\end{itemize}
		\item \textbf{Loss Function:}
		\begin{itemize}
			\item KL-Divergence loss with soft labels from MixUp/CutMix
		\end{itemize}
	\end{itemize}
\end{frame}

% ---------- TRAINING PIPELINE DIAGRAM ----------

\begin{frame}{Training Pipeline of the Proposed Framework}
	\begin{center}
		% Replace the filename with the generated pipeline diagram image
		\includegraphics[width=1\linewidth]{Training_pipeline.png}
	\end{center}
	
\end{frame}
% ---------- FINAL RESULTS ----------
\begin{frame}{Final Performance Results}
	\begin{itemize}
		\item The proposed deep learning framework achieves strong performance on the landslide detection task.
		\item Final evaluation metrics obtained:
	\end{itemize}
	
	\vspace{0.3cm}
	
	\begin{center}
		\begin{tabular}{|l|c|}
			\hline
			\textbf{Metric} & \textbf{Value} \\
			\hline
			Accuracy   & 0.9897 \\
			Precision  & 0.9987 \\
			Recall     & 0.9545 \\
			F1-score   & 0.9761 \\
			Training Loss & 0.1315 \\
			\hline
		\end{tabular}
	\end{center}
	
	\vspace{0.3cm}
	
	\begin{itemize}
		\item High precision indicates very few false positives.
		\item Strong F1-score shows balanced performance on imbalanced data.
		\item Low loss confirms stable and effective model convergence.
	\end{itemize}
\end{frame}

% =========================================================
% DETAILED RESULTS ANALYSIS SLIDES
% (USING GENERATED PERFORMANCE GRAPHS)
% =========================================================

% ---------- ACCURACY VS EPOCH ----------
\begin{frame}{Accuracy vs Epoch}
	\begin{center}
		\includegraphics[width=0.85\linewidth]{accuracy_vs_epoch.png}
	\end{center}
	
	\vspace{0.3cm}
	\begin{itemize}
		\item Shows the progression of model accuracy over training epochs.
		\item Accuracy increases steadily as training progresses.
		\item Indicates effective learning and convergence of the model.
		\item High final accuracy demonstrates strong classification capability.
	\end{itemize}
\end{frame}

% ---------- LOSS VS EPOCH ----------
\begin{frame}{Training Loss vs Epoch}
	\begin{center}
		\includegraphics[width=0.85\linewidth]{loss_vs_epoch.png}
	\end{center}
	
	\vspace{0.3cm}
	\begin{itemize}
		\item Represents the variation of training loss during learning.
		\item Loss decreases consistently with increasing epochs.
		\item Confirms stable optimization and absence of training instability.
		\item Final low loss value indicates good model convergence.
	\end{itemize}
\end{frame}

% ---------- ACCURACY VS LOSS ----------
\begin{frame}{Accuracy vs Loss}
	\begin{center}
		\includegraphics[width=0.85\linewidth]{accuracy_vs_loss.png}
	\end{center}
	
	\vspace{0.3cm}
	\begin{itemize}
		\item Illustrates the inverse relationship between accuracy and loss.
		\item As loss decreases, classification accuracy improves.
		\item Demonstrates consistency between optimization objective and performance.
		\item Validates correctness of the training process.
	\end{itemize}
\end{frame}

% ---------- FINAL METRICS VISUALIZATION ----------
\section{Results}
\begin{frame}{Final Evaluation Metrics}
	\begin{center}
		\includegraphics[width=0.75\linewidth]{final_metrics.png}
	\end{center}
	
	\vspace{0.3cm}
	\begin{itemize}
		\item Summarizes final performance metrics of the proposed model.
		\item High precision indicates very low false positive rate.
		\item Strong recall shows effective detection of landslide regions.
		\item Balanced F1-score confirms robustness on imbalanced data.
	\end{itemize}
\end{frame}

% ---------- PERFORMANCE INTERPRETATION ----------
\begin{frame}{Performance Interpretation}
	\begin{itemize}
		\item Accuracy of \textbf{98.97\%} demonstrates excellent overall performance.
		\item Precision of \textbf{99.87\%} indicates reliable landslide predictions.
		\item Recall of \textbf{95.45\%} confirms effective minority class detection.
		\item Low training loss of \textbf{0.1315} reflects stable convergence.
		\item Results validate the effectiveness of SMOTE and data augmentation.
	\end{itemize}
\end{frame}

% ---------- SUMMARY OF RESULTS ----------
\begin{frame}{Results Summary}
	\begin{itemize}
		\item Proposed deep learning framework achieves high accuracy and robustness.
		\item Class imbalance handling significantly improves minority class performance.
		\item EfficientNetV2 effectively extracts discriminative multispectral features.
		\item The framework is suitable for real-world landslide detection tasks.
	\end{itemize}
\end{frame}
% ---------- CONCLUSION ----------
\section{Conclusion}
\begin{frame}{Conclusion}
	\begin{itemize}
		\item This work presented a deep-learning framework for landslide detection using multispectral satellite imagery.
		\item The proposed approach integrates:
		\begin{itemize}
			\item EfficientNetV2-Large for robust deep feature extraction
			\item Support Vector Machine (SVM) for improved final classification
		\end{itemize}
		\item Class imbalance was effectively addressed using:
		\begin{itemize}
			\item Offline SMOTE-based data augmentation
			\item Online augmentation techniques during training
		\end{itemize}
		\item Extensive experimentation demonstrated that:
		\begin{itemize}
			\item The combination of deep learning, augmentation, and SVM post-processing significantly improves performance
			\item The proposed framework achieves a high F1-score and reliable generalization
		\end{itemize}
		\item Overall, the results confirm the effectiveness of the proposed method for accurate and robust landslide detection in complex environments.
	\end{itemize}
\end{frame}


	
\end{document}